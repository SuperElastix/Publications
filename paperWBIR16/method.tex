\section{Method}
\label{sec:method}
The design we propose
consists of three key elements: a) an
algorithmic network topology that is completely user-configurable, b) a generic component design that is
able to capture a broad diversity of functionalities, and c) a user
configuration that dynamically (without programming) sets up the
network, performs validation via a handshake mechanism, and establishes a connection between
components. In the following sections, we describe the network, how we define components and
how they connect, and how the algorithm is executed.

\subsection{User configurable component networks}

From the analysis of registration algorithms in Section
\ref{sec:analysis} we concluded that although components may be
equal among paradigms, the algorithmic networks can be very
different. Instead of adhering to one algorithmic network (e.g.
\elastix{}) or programmatically hard-code a variety of paradigms
(e.g. ANTs, plastimatch), we make the network topology part of the
user configuration as well. That is, on a high-level the user
describes the network layout in terms of nodes and connections.
Due to the large variety of possible networks we use a light-weighted notion of a network, which is not intrinsically tied to specific functionality of the components at its nodes.
Instead, we choose to define all components to be
handled generically and to dispatch the conceptual validation of the
user configuration to a handshake mechanism, which is explained
next. On a software level the network is modeled as a (boost) graph, where the nodes denote components and the vertices denote connections. 
Examples of such networks are given in Figure \ref{fig:networks}.

\subsection{Component handshakes}

Currently the majority of registration toolboxes adhere a classical
object-oriented design that decomposes the registration problem into
classes like metrics, transforms, optimizers, etc. Extended types of
behavior (mutual information, affine transform, etc) are implemented
via subtyping. However, as we concluded in Section
\ref{sec:analysis}, this decomposition is generally different among
paradigms and toolboxes, and hampers unification. To address the
so-called ``tyranny of the dominant decomposition''
\cite{Tarr:separation}, we introduce a role-based software paradigm,
similar to the Data Context Interaction (DCI) pattern, for
specifying collaboration between components. Building our toolbox
around this design pattern, we are able to reuse code bases without
suffering from the effects of invasive modification and
re-architecture. 
This design allows to cherry-pick behavioral aspects that components
have in common, without a strict classification what that component
is. 

Fundamental to our design is a generic handshake mechanism that
validates whether components can be connected or not. This mechanism
performs the necessary checks on what a component can do, which is
required to establish a connection. The advantage of explicitly
handling this generically and on a higher level, is that components
themselves do not need to perform these checks on neighboring
components, which would require a component to embed specific
knowledge about other components. The proposed design counters the
entanglement between the sets of components of the same code base
and opens up the potential for cross-paradigm connections.

To manage all possible types of collaboration, we maintain an
extensible collection of so-called interfaces between components.
Any component in our toolbox must be defined in terms of one or more
interfaces, which are either \emph{accepting} or \emph{providing}.
Figure \ref{fig:interface} illustrates a component based on various
interfaces. 
The user configures a generic connection between the components, while a handshake mechanism determines the types of interfaces and their compatibility.
If a connection is
possible, the component with the accepting interface gains control
over the communication and is responsible for setting up its
internals for the execution of the registration algorithm.
At the start of the execution of an algorithm all components check if sufficient accepting interfaces have been connected.

From an implementation point of view, all interfaces are defined as
abstract base classes, in the DCI pattern, also known as methodless
Roles. Via helper classes that are variadicly templated, a component
class inherits from any number of interface classes either being
accepting or providing. The developer of a component is responsible
for implementing all chosen interfaces, i.e. methodful Roles in DCI.
In the handshake mechanism compatibility is verified based on the
C++ types of the interfaces. While the number of types of interfaces
may increase with future developments, the handshake mechanism
itself is very general and not that sensitive to changes, due to its
loose coupling to component functionality. As an example, a
handshake mechanism may be in place where image samplers
\emph{provide} a list of samples, while metrics or more aggregated
components \emph{accept} a list of samples. When a new sampling
component becomes available it does not need to inherit from a base
sampling class, which often will require code refactoring, but only
needs to define a providing interface.

Within this design the notion of hybrid components (components that
fulfill multiple tasks) is captured naturally, by simply proving
different accepting or providing interfaces. In this way monolithic
blocks that implement a full registration pipeline can also be
easily integrated in the toolbox.

\begin{figure}
\centering
 \scalebox{1.0}{
  \input{connectors.pdf_tex}
 }
\caption{Component handshakes are performed at run-time based on
interface types. The toolbox maintains an extensible list of
possible interfaces,
e.g.~\protect\includegraphics[height=1.2ex]{IFrect.pdf}
\protect\includegraphics[height=1.2ex]{IFmoon.pdf}
\protect\includegraphics[height=1.2ex]{IFstar.pdf}
\protect\includegraphics[height=1.2ex]{IFtriangle.pdf}
\protect\includegraphics[height=1.2ex]{IFbarbapappa.pdf}
\protect\includegraphics[height=1.2ex]{IFsquare.pdf},
by which components can connect and collaborate.}
\label{fig:interface}
\end{figure}

\subsection{Algorithm embedding}

Whereas previously we described the core functionality of the
toolbox, i.e. the setup of components, the network layout and the
handshake between components, this subsection describes how this
network is embedded in the toolbox, how the algorithm is executed
and how data is passed to and from the toolbox.

For a good deployment of our toolbox it will be provided as a
library. This can then be naturally integrated in many different
software interfaces, such as a command-line application, GUI
applications or scripting languages. Figure \ref{fig:toolbox_layout}
schematically illustrates how the \SuperElastix{} library is
embedded, which incorporates an Overlord and the registration
network. The Configuration Object acts as an intermediate
representation of the user-specified configuration of the
registration algorithm. This contains the network description, the
component names and their settings. In a command-line tool this
configuration is read from disk, whereas in library usage it serves
as a lightweight container that can be manipulated, passed and
stored without requiring disk access.

When a Configuration Object is passed to \SuperElastix{} the
Overlord performs three steps: a) it parses the configuration and
instantiates the network with the required components, providing the
user with feedback on any incompatibilities between components, as detected by the handshake mechanism, b)
it connects the network's Sink and Source components to external
data (pipelines), and c) it executes the registration algorithm by connecting itself, via a handshake, to the component that includes the execution trigger.

The Sink and Source component are illustrated by dashed lines in
Figure \ref{fig:toolbox_layout} and are special types of components
in the sense that the Overlord performs handshakes to these
components to let the data pass. By configuring Sink and Source
components the user controls whether components should create and
pass data such as images, deformation fields or point sets.

\begin{figure}
\centering
 \scalebox{1.0}{
  \input{toolbox_layout.pdf_tex}
 }
\caption{The grand design of the \SuperElastix{} toolbox. The
toolbox can be used as a command-line application or be embedded in
other applications and scripting languages as a library. In this illustration $F$, $M$ and $\phi$ are examples of typical sources and sinks, but the framework supports any type of source and sink.}
\label{fig:toolbox_layout}
\end{figure}
